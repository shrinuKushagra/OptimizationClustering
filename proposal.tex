\documentclass{article}
\usepackage[nonatbib]{nips_2016}

\usepackage[breaklinks=true,letterpaper=true,colorlinks,citecolor=black,bookmarks=false]{hyperref}

%\usepackage{amsthm}
\usepackage{amsmath,amssymb}
\usepackage{enumitem}

\usepackage[sort&compress,numbers]{natbib}
\usepackage[normalem]{ulem}

% use Times
\usepackage{times}
% For figures
\usepackage{graphicx} % more modern
%\usepackage{epsfig} % less modern
%\usepackage{subfig} 

\graphicspath{{../fig/}}

\usepackage{tikz}
\usepackage{tkz-tab}
\usepackage{caption} 
\usepackage{subcaption} 
\usetikzlibrary{shapes.geometric, arrows}
\tikzstyle{arrow} = [very thick,->,>=stealth]

\usepackage{cleveref}
\usepackage{setspace}
\usepackage{wrapfig}
%\usepackage[ruled]{algorithm}
\usepackage{algpseudocode}
%\usepackage[noend,linesnumbered]{algorithm2e}

\usepackage[disable]{todonotes}


\title{Proposal for CS798, Fall 2016\\ \large Optimization for Machine Learning}

\author{
	Nicole McNabb \\
	School of Computer Science\\
	University of Waterloo\\
	Waterloo, ON, N2L 3G1 \\
	\texttt{nmcnabb@uwaterloo.ca} \\
	\And
	Shrinu Kushagra\\
	School of Computer Science\\
	University of Waterloo\\
	Waterloo, ON, N2L 3G1 \\
	\texttt{skushagr@uwaterloo.ca} \\	
}

\begin{document}
\maketitle

\begin{abstract} 
Put here a brief summary of the project: what is it about? what is the main goal? how are you going to evaluate it? The proposal is expected to be 1-2 pages, so be concise and to the point.
\end{abstract} 

\section{Introduction}
In this section you are going to present a brief background and motivation about your project. Why is it interesting/significant? How does it relate to the course?

Clustering is a technique of unsupervised learning that attempts to group data points that are similar in features into "clusters". It can be used both as a tool to understand the underlying distibution of data or as a preprocessing step to alleviate the hardness of classification. While efficient clustering methods such as $k$-means are widely used in practice, many clustering objectives are NP-hard to optimize over all possible data sets. Because of this, a primary aim of recent clustering research is to show instead that there exist  clustering algorithms that perform efficiently over the set of all data sets that are likely to be seen in practice. [This is informally referred to as the "Clustering is difficult only when it does not matter" thesis.] In order to 

\section{Related Works}
Perform an initial review of relevant literature. Has your problem, or one of similar nature, been considered before? By whom? What are the differences or limitations (if any)? 

\section{Proposed Work}
In this section please concisely describe what you are going to achieve in this project. E.g., formulate the mathematical problem under consideration, present the technical challenges (if any), discuss the tools or datasets that you will build on, state your goals, and come up with a plan for evaluation.

For your own sake, you might want to lay out a time line, so that you can keep a good track of your project.

\section{Team}
If you have teammates, state here the different roles of each person. Please be aware that each teammate will get the \emph{same} evaluation, so choose your teammate wisely. \uline{No excuse or regret once the team is formed}.


\section*{Acknowledgement}
Thank people who have helped or influenced you in this project.

\bibliographystyle{unsrtnat}
\bibliography{proposal}

\end{document}